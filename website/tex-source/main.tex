\documentclass[12pt, a4paper]{article}

% 中文支持
\usepackage[UTF8]{ctex}

% 数学符号和定理环境
\usepackage{amssymb, amsmath, amsthm}

% 页面布局和行间距
\usepackage[top=1in, bottom=1in, left=1in, right=1in]{geometry}
\usepackage{setspace}
\setstretch{1.3} % 行间距

% 定理环境设置
\newtheorem{theorem}{定理}[section] % 定理按章节编号
\newtheorem{lemma}[theorem]{引理}
\newtheorem{corollary}[theorem]{推论}
\newtheorem{proposition}[theorem]{命题}
\theoremstyle{definition}
\newtheorem{definition}[theorem]{定义}
\newtheorem{example}[theorem]{例子}
\theoremstyle{remark}
\newtheorem{remark}[theorem]{注释}

% 证明环境
\newenvironment{solution}
  {\begin{proof}[\textbf{解答}]}
  {\end{proof}}

% 超链接设置
\usepackage{hyperref}
\hypersetup{
    colorlinks=true,
    linkcolor=blue,
    urlcolor=blue,
    citecolor=blue
}

% 页眉和页脚
\usepackage{fancyhdr}
\pagestyle{fancy}
\fancyhf{}
\fancyhead[L]{2025 Awesome Math} % 页眉左侧内容
\fancyhead[R]{\thepage} % 页眉右侧显示页码
\fancyfoot[C]{\textit{Part II}}

% 数学符号增强
\usepackage{mathrsfs} % 花体字母
\usepackage{bm} % 加粗数学符号

% 颜色
\usepackage{xcolor}

\begin{document}

\title{\textbf{2025 AWESOME A Part II Solution}}
\author{作者:TALENT}
\date{\today}
\maketitle

\tableofcontents % 目录
\newpage

\section{第一题}

\noindent 设 $a_1>0$,且满足递推关系

$$
a_{n+1}=a_1 a_2 \cdot \ldots \cdot a_n+1, \quad n=1,2,3, \ldots
$$

\noindent 求整数 $m$,使得

$$
\sqrt{a_2-\frac{3}{4}}+\sqrt{a_3-\frac{3}{4}}+\cdots+\sqrt{a_{2025}-\frac{3}{4}}=a_1+a_2+\cdots+a_{2024}+m
$$


\subsection{解答}

\noindent 在处理形如
$$
a_{n+1}=\underbrace{a_1 a_2 \cdots a_n}_{P_n}+1
$$
的递推时,一个常见的技巧是观察其是否与某些“带平方根”的恒等式产生联系,例如 $\sqrt{x-\frac{3}{4}}$ 的形式。

\noindent 因此,我们猜测以下引理成立,并尝试证明:
\begin{lemma}
若递推关系为
$$
a_{n+1}=a_1 a_2 \cdots a_n+1, \quad n=1,2,3,\ldots
$$
则有
$$
\sqrt{a_{n+1}-\frac{3}{4}} = a_n-\frac{1}{2}.
$$
\end{lemma}

\begin{proof}

\noindent 首先验证 $n=2$ 的特殊情况是否满足。

\noindent 根据递推关系,有
$$
a_2=a_1+1, \quad a_3=a_1 a_2+1=a_1\left(a_1+1\right)+1=a_1^2+a_1+1.
$$
计算左边:
$$
\sqrt{a_3-\frac{3}{4}}=\sqrt{\left(a_1^2+a_1+1\right)-\frac{3}{4}}=\sqrt{a_1^2+a_1+\frac{1}{4}}.
$$
注意到 $a_1^2+a_1+\frac{1}{4}=\left(a_1+\frac{1}{2}\right)^2$,因此
$$
\sqrt{a_1^2+a_1+\frac{1}{4}}=a_1+\frac{1}{2}, \quad (\text{取正根, 因 } a_1>0).
$$
计算右边:
$$
a_2-\frac{1}{2}=\left(a_1+1\right)-\frac{1}{2}=a_1+\frac{1}{2}.
$$
左边与右边相等,故
$$
\sqrt{a_3-\frac{3}{4}}=a_2-\frac{1}{2}, \quad (n=2).
$$

\noindent 接下来验证一般情形 $n \geq 2$。

\noindent 假设对于某个 $n \geq 2$,已知
$$
\sqrt{a_{n+1}-\frac{3}{4}}=a_n-\frac{1}{2}.
$$
我们需要证明
$$
\sqrt{a_{n+2}-\frac{3}{4}}=a_{n+1}-\frac{1}{2}.
$$
由递推关系,有
$$
a_{n+2}=a_1 a_2 \cdots a_{n+1}+1.
$$
因此,
$$
a_{n+2}-\frac{3}{4}=\left(a_1 a_2 \cdots a_{n+1}\right)+\frac{1}{4}.
$$
我们希望其平方根满足
$$
\sqrt{a_{n+2}-\frac{3}{4}}=a_{n+1}-\frac{1}{2}.
$$
为此,只需证明
$$
\left(a_{n+1}-\frac{1}{2}\right)^2=\left(a_1 a_2 \cdots a_{n+1}\right)+\frac{1}{4}.
$$
展开左边:
$$
\left(a_{n+1}-\frac{1}{2}\right)^2=a_{n+1}^2-a_{n+1}+\frac{1}{4}.
$$
计算右边。由递推关系,$a_1 a_2 \cdots a_n = a_{n+1}-1$,因此
$$
a_1 a_2 \cdots a_{n+1}=a_{n+1}\left(a_{n+1}-1\right)=a_{n+1}^2-a_{n+1}.
$$
将 $\frac{1}{4}$ 加入,得
$$
a_1 a_2 \cdots a_{n+1}+\frac{1}{4}=a_{n+1}^2-a_{n+1}+\frac{1}{4}.
$$
左边与右边一致,因此
$$
\sqrt{a_{n+2}-\frac{3}{4}}=a_{n+1}-\frac{1}{2}.
$$

\noindent 综上,归纳法表明,对于 $n \geq 2$,
$$
\sqrt{a_{n+1}-\frac{3}{4}} = a_n-\frac{1}{2}.
$$
\end{proof}

\noindent 注意:当 $n=1$ 时,
$$
\sqrt{a_2-\frac{3}{4}}=\sqrt{a_1+1-\frac{3}{4}}=\sqrt{a_1+\frac{1}{4}}.
$$
这通常不等于 $a_1-\frac{1}{2}$(除非 $a_1=2$ 恰好成立)。因此,上述结论仅适用于 $n \geq 2$。

\noindent 基于上述结论,原题可以改写为
$$
\sum_{k=2}^{2025} \sqrt{a_k-\frac{3}{4}}=\underbrace{\sqrt{a_2-\frac{3}{4}}}_{\text{需单独保留}}+\sum_{k=3}^{2025} \sqrt{a_k-\frac{3}{4}}.
$$
对于首项:
$$
\sqrt{a_2-\frac{3}{4}}=\sqrt{a_1+1-\frac{3}{4}}=\sqrt{a_1+\frac{1}{4}}.
$$
对于后续项,根据引理,
$$
\sum_{k=3}^{2025} \sqrt{a_k-\frac{3}{4}}=\sum_{k=3}^{2025}\left(a_{k-1}-\frac{1}{2}\right)=\sum_{j=2}^{2024}\left(a_j-\frac{1}{2}\right),
$$
(将索引替换为 $j=k-1$)。

\noindent 合并可得
$$
\sum_{k=2}^{2025} \sqrt{a_k-\frac{3}{4}}=\sqrt{a_1+\frac{1}{4}}+\left(\sum_{j=2}^{2024} a_j\right)-\frac{2023}{2}.
$$
进一步利用
$$
\sum_{j=2}^{2024} a_j=\sum_{j=1}^{2024} a_j-a_1,
$$
得到
$$
\sum_{k=2}^{2025} \sqrt{a_k-\frac{3}{4}}=\sum_{j=1}^{2024} a_j-a_1-\frac{2023}{2}+\sqrt{a_1+\frac{1}{4}}.
$$
题目右边为
$$
\sum_{j=1}^{2024} a_j+m.
$$
比较两边,消去公共项 $\sum_{j=1}^{2024} a_j$,得
$$
m=-a_1-\frac{2023}{2}+\sqrt{a_1+\frac{1}{4}}.
$$
即
$$
m=\sqrt{a_1+\frac{1}{4}}-a_1-1011.5.
$$
注意到当 $a_1=2$ 时,有
$$
m=\sqrt{2+\frac{1}{4}}-2-1011.5=1.5-2-1011.5=-1012.
$$

\newpage

\section{第二题}

\noindent 已知

$$
\prod_{n=5}^{2025}\left(\frac{n}{2}-2+\frac{3 n+1}{n^2+n+1}\right)=\frac{k}{2^m} \cdot \frac{2023!}{2025 \cdot 2026+1}
$$

\noindent 其中 $k$ 和 $m$ 是正整数,且 $k$ 是奇数,求 $10k+m$。

\subsection{解答}

\noindent 化简等式左边可以得到:
$$
\prod_{n=5}^{2025} \left( \frac{n}{2} - 2 + \frac{3n+1}{n^2+n+1} \right)
= \prod_{n=5}^{2025} \left( \frac{n-4}{2} + \frac{3n+1}{n^2+n+1} \right)
$$
$$
= \prod_{n=5}^{2025} \frac{(n-4)(n^2+n+1) + 2(3n+1)}{2(n^2+n+1)}
$$
$$
= \prod_{n=5}^{2025} \frac{n^3 + n^2 + n - 4n^2 - 4n - 4 + 6n + 2}{2(n^2+n+1)} 
$$
$$
= \prod_{n=5}^{2025} \frac{n^3 - 3n^2 + 3n - 2}{2(n^2+n+1)}.
$$

\noindent 我们注意到
$$
n^3 -3n^2 +3n -2 \;=\; (n-1)^3 \;-\;1 
\;=\; (n-1)^3 - 1^3.
$$
再用立方差分解式
$$
a^3 - b^3 = (a - b)\,(a^2 + ab + b^2),
$$
令 \(a=n-1,\; b=1\),则
$$
(n-1)^3 - 1^3 
= \left((n-1)-1\right)\left((n-1)^2 + (n-1)\cdot 1 + 1^2\right) 
= (n-2)\,\left(n^2 -n +1\right).
$$
这样,原表达式即化为
$$
\prod_{n=5}^{2025} \frac{(n-2)\,(n^2 - n +1)}{2\,(n^2 + n +1)} 
\;=\; \prod_{n=5}^{2025} \left( \frac{n-2}{2}\;\cdot\;\frac{n^2 - n +1}{n^2 + n +1} \right).
$$

\noindent 对于第一部分:
$$
\prod_{n=5}^{2025} \frac{n-2}{2}
$$
共有 \(2025 - 5 + 1 = 2021\) 个因子,每个因子分母都有一个 2,故分母部分贡献 \(2^{2021}\)。

\noindent 分子部分是 \((n-2)\) 从 \(n=5\) 到 \(n=2025\) 的连乘积,即
$$
\prod_{n=5}^{2025}(n-2) 
= 3 \times 4 \times 5 \times \cdots \times 2023 = \frac{2023!}{2}.
$$
所以
$$
\prod_{n=5}^{2025} \frac{n-2}{2}
= \frac{3\cdots 2023}{\,2^{2021}} 
= \frac{\frac{2023!}{2}}{\,2^{2021}}
= \frac{2023!}{\,2^{2022}}.
$$

\noindent 对于第二部分:
$$
\prod_{n=5}^{2025} \frac{n^2 - n +1}{n^2 + n +1}
$$
记
$$
a_n = n^2 - n + 1.
$$
注意到
$$
a_{n+1} 
= (n+1)^2 - (n+1) + 1 
= n^2 + 2n +1 \;-\; n -1 \;+\;1 
= n^2 + n +1.
$$
恰好是分母 \(n^2 + n +1\)。因此
$$
\frac{n^2 - n +1}{n^2 + n +1} 
= \frac{a_n}{a_{n+1}}.
$$
这样,从 \(n=5\) 到 \(2025\) 的连乘就望去是一个(Telescoping) 形式:
$$
\prod_{n=5}^{2025} \frac{a_n}{a_{n+1}}
= \frac{a_5}{a_{2026}}.
$$
容易计算:
$$
a_5 = 5^2 - 5 + 1 = 25 - 5 +1 = 21,
$$
$$
a_{2026} 
= 2026^2 \;-\;2026 \;+\;1.
$$
$$
a_{2026} = 2025 \times 2026 + 1.
$$
因此
$$
\prod_{n=5}^{2025}\frac{n^2 - n +1}{n^2 + n +1}
= \frac{21}{\,2025 \times 2026 +1}.
$$

\noindent 所以
$$
\prod_{n=5}^{2025}\left(\tfrac{n}{2} -2 + \tfrac{3n+1}{n^2+n+1}\right)
=\left(\frac{2023!}{\,2^{2022}}\right)
\;\times\;
\left(\frac{21}{\,2025\cdot 2026 +1}\right)
= \frac{\,21 \cdot 2023!}{\,2^{2022}\,\left(2025\cdot 2026 +1\right)}.
$$

\noindent 由题
$$
\prod_{n=5}^{2025}\left(\tfrac{n}{2} -2 + \tfrac{3n+1}{n^2+n+1}\right) = \frac{k}{2^m}\cdot \frac{2023!}{\,2025\cdot 2026+1}.
$$
其中 \(k\) 为正奇数,\(m\) 为正整数。

\noindent 比较可知
$$
k = 21,\quad m = 2022.
$$

\noindent 计算
$$
10k + m = 10 \times 21 + 2022 = 210 + 2022 = 2232.
$$

\newpage

\section{第三题}

\noindent 解方程:

$$
\sqrt{x^2+45x+2025}-\sqrt{x^2-24x+576}=39。
$$

\subsection{解答}

\noindent 令
$$
A \;=\;\sqrt{x^2 + 45x + 2025}, 
\quad
B \;=\;\sqrt{x^2 - 24x + 576}.
$$
題目給出
$$
A - B \;=\;39
\;\;\Rightarrow\;\;
A \;=\;39 + B.
$$
$$
A^2 \;=\;(39 + B)^2 
\;\;\Rightarrow\;\;
x^2 + 45x + 2025 
\;=\;
39^2 \;+\; 78\,B \;+\; B^2.
$$
因此代入可得
$$
x^2 + 45x + 2025 
\;=\;
39^2 
\;+\;
78\,\sqrt{x^2 - 24x + 576}
\;+\;
\left(x^2 - 24x + 576\right).
$$
$$
x^2 + 45x + 2025
\;=\;
x^2 - 24x + 576
\;+\;
1521
\;+\;
78\,\sqrt{x^2 - 24x + 576}.
$$
$$
45x + 24x + 2025 - 2097
\;=\;
78\,\sqrt{x^2 - 24x + 576}.
$$
$$
69x - 72 
\;=\;
78\,\sqrt{x^2 - 24x + 576}.
$$

\noindent 因此
$$
\sqrt{x^2 - 24x + 576}
\;=\;
\dfrac{69x - 72}{78}.
$$
(注意右边必须满足 \(\ge 0\),亦即 \(69x - 72 \ge 0\),从而 \(x \ge \tfrac{72}{69} = \tfrac{24}{23}\)。)

$$
x^2 - 24x + 576
\;=\;
\left(\tfrac{69x - 72}{78}\right)^2.
$$
$$
6084\,\left(x^2 - 24x + 576\right)
\;=\;
(69x - 72)^2.
$$

\noindent 现在我们展开两边并进行简单的计算:
$$
6084\,x^2 
\;-\;
6084 \cdot 24\,x
\;+\;
6084 \cdot 576.
$$

\noindent 由于\(6084 \cdot 24 = 146016\),\, \(6084 \cdot 576 = 3{,}504{,}384.\)

\noindent 所以左边为
$$
6084\,x^2 
\;-\; 146016\,x 
\;+\; 3{,}504{,}384.
$$

\noindent 右边为
$$
(69x - 72)^2 
\;=\;
(69x)^2
\;-\;
2 \cdot 69x \cdot 72
\;+\;
72^2.
$$
由于\( (69x)^2 = 4761\,x^2\), \, \( 2 \cdot 69 \cdot 72 \,=\, 9936\), \, \( 72^2 = 5184.\)

\noindent 所以右边为
$$
4761\,x^2 \;-\; 9936\,x \;+\; 5184.
$$

\noindent 合并两边可得
$$
6084\,x^2 - 146016\,x + 3{,}504{,}384 
\;-\;
\left(4761\,x^2 - 9936\,x + 5184\right)
\;=\;0.
$$
$$
1323\,x^2 
\;-\;
136080\,x 
\;+\;
3{,}499{,}200
\;=\;
0.
$$

\noindent 注意到所有系数都有因子27,所以
$$
27
\left(49\,x^2 \;-\;5040\,x \;+\;129600\right)
\;=\;0
\;\;\Rightarrow\;\;
49\,x^2 \;-\;5040\,x \;+\;129600
\;=\;0.
$$
$$
x \;=\;
\dfrac{5040 \;\pm\; \sqrt{(5040)^2 \;-\;4\cdot 49 \cdot 129600}}{2\cdot 49}.
$$
注意到判别式恰好为0,只有一个实根:
$$
x 
\;=\;
\dfrac{5040}{2 \cdot 49}
\;=\;
\dfrac{2520}{49}
\;=\;
\dfrac{360}{7}.
$$

\noindent 显然\(\dfrac{360}{7} \gg \frac{24}{23} \approx 1\),故其符合我们前面对于域的要求。

\newpage

\section{第四题}

\noindent 找出一个合适的 $n$ 和符号的选择(包含 "+" 和 "-",且至少包含一个减号),使得:

$$
\pm 1^3 \pm 2^3 \pm \cdots \pm n^3=2025
$$


\subsection{解答}

\noindent 我们知道著名恒等式:
$$
1^3+2^3+3^3+\cdots+n^3=\left(\frac{n(n+1)}{2}\right)^2
$$

\noindent 且我们知道,
$$
2025 = 45^2 = \left(\frac{9 \times 10}{2}\right)^2 = \sum_{k=1}^9 k^3
$$

\noindent 因此,$n=9$ 是一个最小的可能满足的 $n$,虽然其不能直接给出符合题目要求的答案,但给了我们需要尝试的 $n$ 的最小界。

\noindent 如果我们令某些 $k^3$ 的符号都为负,那么其对总和的影响是减去 $2 k^3$ 。也就是说,如果
$$
S_n=1^3+2^3+\cdots+n^3=\left(\frac{n(n+1)}{2}\right)^2,
$$
那么带符号的和可写为
$$
\underbrace{\left(+1^3+\cdots+n^3\right)}_{\text{全正号和 } S_n} - 2 \times \text{(被翻为负号的那部分立方和)}.
$$

\noindent 我们想要
$$
S_n - 2 \times \text{(负号部分立方和)} = 2025,
$$
从而得到
$$
\text{负号部分立方和} = \frac{S_n - 2025}{2}.
$$

\noindent 所以可以发现如下限制:
\begin{itemize}
    \item $S_n - 2025 \geq 0$,即 $S_n \geq 2025$,否则没法把和降到 2025。
    \item $S_n - 2025$ 必须是偶数,否则 $\frac{S_n - 2025}{2}$ 不会是整数,翻号和就凑不成 2025。
    \item $\frac{S_n - 2025}{2}$ 只能由若干互不相同的 $k^3$ 之和组成(因为每个整数 $k$ 只能出现一次,其立方也只能出现一次)。
\end{itemize}

\noindent 我们紧接着尝试几项来帮助找到更多的规律。

\begin{itemize}
    \item 对于 $n = 10$,
    $$
    S_{10} = 1^3 + 2^3 + \cdots + 10^3 = \left( \frac{10 \times 11}{2} \right)^2 = 55^2 = 3025.
    $$
    $$
    S_{10} - 2025 = 1000 \text{ 是偶数,} \quad \frac{S_{10} - 2025}{2} = 500.
    $$
    需在 $\{1, 8, 27, 64, 125, 216, 343, 512, 729, 1000\}$ 中找到子集和为 500 的组合,但无法满足。因此 $n = 10$ 无解。

    \item 对于 $n = 11$,
    $$
    S_{11} = \left( \frac{11 \times 12}{2} \right)^2 = 66^2 = 4356.
    $$
    $$
    S_{11} - 2025 = 2331 \text{ 是奇数,} \quad \frac{S_{11} - 2025}{2} = 1165.5 \text{ 不是整数,不满足限制条件。}
    $$

    \item 对于 $n = 12$,
    $$
    S_{12} = 78^2 = 6084.
    $$
    $$
    S_{12} - 2025 = 4059 \text{ 依然是奇数,} \quad \frac{S_{12} - 2025}{2} = 2029.5 \text{ 不是整数,不满足限制条件。}
    $$

    \item 对于 $n = 13$,
    $$
    S_{13} = 91^2 = 8281.
    $$
    $$
    S_{13} - 2025 = 6256 \text{ 是偶数,} \quad \frac{S_{13} - 2025}{2} = 3128.
    $$
    需在 $\{1, 8, 27, 64, 125, 216, 343, 512, 729, 1000, 1331, 1728, 2197\}$ 找到子集和为 3128 的组合,但无法满足。因此 $n = 13$ 无解。

    \item 对于 $n=14$,
    $$
    S_{14}=\left(\frac{14 \times 15}{2}\right)^2=105^2=11025.
    $$
    $$
    S_{14} - 2025 = 11025 - 2025 = 9000, \quad \frac{S_{14} - 2025}{2} = 4500.
    $$
    需在 $\{1^3, 2^3, \ldots, 14^3\}$ 找到子集和为 4500 的组合。

    \noindent 由于 4500 大于任意单独子集,我们从 $14^3$ 开始向下计算组合可能。
    $$
    4500 - 14^3 = 4500 - 2744 = 1756.
    $$
    注意到 $12^3$ 接近于 1756,于是我们先尝试,
    $$
    1756 - 12^3 = 1756 - 1728 = 28.
    $$
    注意到 $28 = 1^3 + 3^3$,所以我们可以找到 $n = 14$ 时满足子集合为 4500 的情况:
    $$
    4500 = 14^3 + 12^3 + 3^3 + 1^3.
    $$
\end{itemize}


\newpage

\section{第五题}

\noindent 给出一个实数 \(a\) ,满足:
\[
2025a^2 + \frac{1}{45a} = 45a
\]

\noindent 计算:
\[
2025^2a^4 + \frac{1}{45a} 
\]


\subsection{解答}

\noindent 令
$$
p \;=\; 45\,a.
$$
那么
$$
2025 \;=\; 45^2,
\quad
2025\,a^2 \;=\; 45^2\,a^2 \;=\; (45\,a)^2 \;=\; p^2.
$$

$$
2025\,a^2 \;+\;\frac{1}{45\,a} \;=\; 45\,a \;\Rightarrow\; p^2 \;+\;\frac{1}{p} \;=\; p.
$$

\noindent 注意到
$$
2025^2\,a^4 
\;=\;
(2025\,a^2)^2 
\;=\;
(p^2)^2 
\;=\;
p^4,
$$
并且
$$
\frac{1}{45\,a} \;=\;\frac{1}{p}.
$$
因此所求等价于
$$
p^4 \;+\;\frac{1}{p}.
$$

\noindent 于是,原问题化为:

已知 \(p\) 满足
$$
p^2 \;+\;\frac{1}{p} \;=\; p,
$$
试求
$$
p^4 \;+\;\frac{1}{p}.
$$

\noindent 整理可得
$$
p^2 \;-\; p \;+\;\frac{1}{p} \;=\; 0.
$$
$$
p^3 \;-\; p^2 \;+\; 1 \;=\; 0,
\quad
p^3 \;=\; p^2 \;-\; 1.
$$
$$
p^4 
\;=\; p \cdot p^3 
\;=\; p\,\bigl(p^2 \;-\; 1\bigr)
\;=\; p^3 \;-\; p
\;=\; (p^2 - 1) \;-\; p
\;=\; p^2 \;-\; p \;-\; 1.
$$

\noindent 另外从原方程
$$
p^2 + \frac{1}{p} = p
$$
可知
$$
\frac{1}{p} \;=\; p \;-\; p^2.
$$
因此
$$
p^4 \;+\;\frac{1}{p}
\;=\;
\bigl(p^2 \;-\; p \;-\; 1\bigr)
\;+\;
\bigl(p \;-\; p^2\bigr)
\;=\;
-1.
$$

\noindent 所以
$$
2025^2\,a^4 \;+\;\frac{1}{45\,a}
\;=\;
p^4 + \frac{1}{p}
\;=\;
-1.
$$

\newpage

\section{第六题}

\noindent 满足方程 
$$
44a^2 = 2025\lfloor a\rfloor \{a\}
$$ 

\noindent 的最大实数 $a$ 可以表示为 $\frac{m}{n}$,其中 $m$ 和 $n$ 是互质的正整数。求 $m+n$ 的值。

\noindent 这里,$\lfloor a\rfloor$ 表示 $a$ 的向下取整(不超过 $a$ 的最大整数),$\{a\}$ 表示 $a$ 的小数部分。


\subsection{解答}

\noindent 令
$$
n \;=\; \lfloor a\rfloor \quad(\text{整数}), 
\quad x \;=\; \{a\} \;=\; a - n \quad (0 \le x < 1).
$$
那么 \(a = n + x\),方程变为
$$
44\, (n + x)^2 \;=\; 2025\,n\,x.
$$
$$
44\,(n^2 + 2nx + x^2)
\;=\;
2025\,n\,x.
$$
$$
44\,n^2 \;+\; 88\,n\,x \;+\; 44\,x^2
\;=\;
2025\,n\,x.
$$
$$
44\,n^2 
\;+\; (88 - 2025)\,n\,x
\;+\;
44\,x^2
\;=\;
0.
$$
$$
44\,n^2 
\;-\; 1937\,n\,x
\;+\;
44\,x^2
\;=\;
0.
$$
\noindent 判别式为
$$
\Delta 
\;=\;
(-1937\,n)^2 \;-\; 4\,\cdot\,44\,\cdot\,44\,n^2
\;=\;
n^2\bigl(1937^2 \;-\; 4\cdot 44\cdot 44\bigr).
$$
\noindent 因此方程的根为
$$
x \;=\;
\frac{1937\,n \;\pm\; n \cdot 1935}{2 \cdot 44}
\;=\;
\frac{n}{88}\,\bigl(1937 \pm 1935\bigr).
$$
\noindent 故
$$
x_1 \;=\; \frac{n}{88}\,(1937 + 1935) \;=\; \frac{n}{88}\,\cdot 3872 = 44n.
$$
$$
x_2 \;=\; \frac{n}{88}\,(1937 - 1935) \;=\; \frac{n}{88}\,\cdot 2 = \frac{n}{44}.
$$
\noindent 由于 \(x < 1\) 为小数部分,分析可知:
\begin{itemize}
    \item 对 \(x_1\),若 \(n\) 为正整数则 \(44n \ge 44\),显然 \(x_1 \ge 1\),不满足 \(x<1\)。  
    \item 对 \(x_2\),要满足 \(0 \le x_2 < 1\),则需 \(0 \le n < 44\)。因为 \(n\) 是整数,故 \(n\) 最大可取 \(43\)。  
\end{itemize}

\noindent 于是
$$
x \;=\; \frac{n}{44},
\quad
a \;=\; n + x \;=\; n + \frac{n}{44} \;=\; n\Bigl(1 + \frac{1}{44}\Bigr) \;=\; \frac{45n}{44}.
$$
在 \(n = 1,2,3,\dots,43\) 中,为使 \(a\) 最大,取 \(n=43\)。  

\noindent 此时
$$
a \;=\; \frac{45 \cdot 43}{44} \;=\; \frac{1935}{44}.
$$
\noindent 这里
$$
m = 1935,
\quad
n = 44,
$$
于是
$$
m + n 
\;=\;
1935 \;+\; 44
\;=\;
1979.
$$

\newpage

\section{第七题}

\noindent 求解以下方程组在整数解中的解:

$$
\begin{aligned}
x^3 - 3xyz + y^3 & = 172, \\
x^2 - xy + y^2 + \frac{2025}{x + y + z} & = z(x + y - z).
\end{aligned}
$$



\subsection{解答}

\noindent 令
$$
S \;=\; x + y, 
\quad
M \;=\; x + y + z,
\quad
P \;=\; x\,y.
$$

\noindent 对于式一
$$
x^3 \;+\; y^3 \;-\; 3\,x\,y\,z \;=\; 172,
$$
可以写成
$$
x^3 + y^3 \;=\; 172 \;+\; 3\,x\,y\,z.
$$

\noindent 因为
$$
x^3 + y^3 
\;=\;
(x+y)\,(x^2 - x\,y + y^2)
\;=\;
S\,(x^2 - x\,y + y^2),
$$
且
$$
x^2 - x\,y + y^2 
\;=\;
(x+y)^2 - 3\,x\,y 
\;=\;
S^2 \;-\;3\,P,
$$
所以式一可以写成
$$
S \,\bigl(S^2 - 3\,P \bigr)
\;-\;
3\,P\,z
\;=\;
172.
$$

\noindent 因为 \(z = M - S\),有
$$
\underbrace{S \,\bigl(S^2 - 3\,P \bigr)}_{=\,S^3 - 3\,P\,S}
\;-\;
3\,P\,\bigl(M - S \bigr)
\;=\;
172.
$$
整理可得
$$
S^3 
\;-\;
3\,P\,S
\;-\;
3\,P\,M
\;+\;
3\,P\,S
\;=\;
172
\;\Longrightarrow\;
S^3 
\;-\;
3\,P\,M
\;=\;
172.
$$
即
\[
3\,P\,M 
\;=\;
S^3 - 172
\;\;\Longrightarrow\;\;
P\,M
\;=\;
\dfrac{S^3 - 172}{\,3\,}.
\tag{1'}
\]

\noindent 对于式二
$$
x^2 - x\,y + y^2 \;+\;\dfrac{2025}{\,x + y + z\,} 
\;=\;
z\,(x+y - z),
$$
且 
$$
x^2 - x\,y + y^2
\;=\;
S^2 - 3\,P,
$$
则左边是
$$
\bigl(S^2 - 3\,P\bigr) + \dfrac{2025}{\,M\,},
$$
右边是
$$
(M-S)\,\bigl(S - (M-S)\bigr)
\;=\;
(M-S)\,(2\,S - M).
$$
于是我们有
$$
S^2 - 3\,P + \dfrac{2025}{\,M\,}
\;=\;
(M - S)\,\bigl(2\,S - M\bigr).
$$

\noindent 左边乘上 \(M\) 得
$$
M\,S^2 \;-\; 3\,P\,M \;+\; 2025
\;=\;
M\,(M - S)\,(2\,S - M).
$$
将 (1') 代入可得
$$
M\,S^2 \;-\; \bigl(S^3 - 172\bigr) + 2025
\;=\;
M\,(M - S)\,(2\,S - M).
$$
左边化简为
$$
M\,S^2 \;-\; S^3 + 172 + 2025
\;=\;
M\,S^2 \;-\; S^3 + 13^3,
$$
右边化简为
$$
M\,(M-S)\,(2S - M)
\;=\;
-\,2\,M\,S^2 \;+\;3\,M^2\,S \;-\;M^3.
$$
因此
$$
M\,S^2 - S^3 + 2197
\;=\;
-\,2\,M\,S^2 + 3\,M^2\,S - M^3.
$$
整理得
$$
0
\;=\;
-\,3\,M\,S^2 
\;+\;
3\,M^2\,S
\;-\;
M^3
\;+\;
S^3
\;-\;
13^3.
$$

\noindent 注意到,
$$
(S - M)^3 
\;=\;
S^3
\;-\;3\,S^2\,M
\;+\;3\,S\,M^2
\;-\;M^3.
$$
于是可以写成
\[
(S-M)^3 \;-\;2197 \;=\;0
\;\;\Longrightarrow\;\;
(S-M)^3 \;=\;13^3
\;\;\Longrightarrow\;\;
S - M 
\;=\;
13.
\]
所以
\[
M 
\;=\;
S - 13.
\tag{2'}
\]

\noindent 把 \(M = S - 13\) 代回式一,从 (1') 可知
$$
P\,M 
\;=\;
\dfrac{S^3 - 172}{\,3\,}.
$$
而 \(M = S - 13\),因此
$$
P 
\;=\;
\dfrac{\,S^3 - 172\,}{\,3\,(S - 13)\!}.
$$

\noindent 为使 \(P = x\,y\) 为整数,必须让
$$
\dfrac{S^3 - 172}{\,S - 13\,}
$$
能被 \(3\) 整除。

\noindent 注意到
$$
S^3 - 172 
\;=\;
(S - 13)\,\Bigl(S^2 + 13\,S + 169\Bigr)
\;+\;
2025.
$$
所以
$$
\dfrac{S^3 - 172}{\,S - 13\,}
\;=\;
S^2 + 13\,S + 169 
\;+\;
\dfrac{2025}{\,S - 13\,}.
$$
进而
\[
P
\;=\;
\dfrac{\,S^3 - 172\,}{\,3\,(S - 13)\!}
\;=\;
\dfrac{S^2 + 13\,S + 169}{\,3\,}
\;+\;
\dfrac{675}{\,S - 13\,}.
\tag{3}
\]

\noindent 为满足 \(P\) 为整数,我们有如下限制:
\begin{itemize}
    \item $\frac{S^2+13 S+169}{3}$ 为整数;
    \item $\frac{675}{S-13}$ 为整数。
\end{itemize}

\noindent 经过筛选,当 \(S = 4\) 时,满足所有限制。此时
$$
M = S - 13 = -9.
$$
代入得
$$
P = 4.
$$
解方程
$$
x + y = 4, \quad x\,y = 4,
$$
得到唯一解 \(x = 2, \, y = 2\),且
$$
z = M - S = -9 - 4 = -13.
$$
因此,唯一的整数三元组为
$$
(x, y, z) = (2, 2, -13).
$$

\newpage

\section{第八题}

\noindent 考虑数列 $1, 9, 17, 25, \ldots, 2025$。我们选择两个数 $x$ 和 $y$,并用数 $x + y - 1$ 替换它们。经过 253 次这样的操作后,剩下的数是多少?


\subsection{解答}

\noindent 观察到它是一个首项 \(a_1 = 1\)、公差 \(d = 8\) 的等差数列。  
若最后一项为 \(2025\),则可由一般项公式:
$$
a_n = a_1 + (n - 1)d
$$
求出 \(n\):
$$
2025 = 1 + (n - 1)\times 8
\quad\Longrightarrow\quad
2024 = 8(n - 1)
\quad\Longrightarrow\quad
n - 1 = 253
\quad\Longrightarrow\quad
n = 254.
$$

\noindent 因此,原本共有 \(254\) 个数字。

\noindent 由等差数列求和公式:
$$
S = \frac{n}{2}\bigl(a_1 + a_n\bigr) 
   = \frac{254}{2} \times (1 + 2025) 
   = 127 \times 2026.
$$
计算得:
$$
S = 2026 \times (100 + 27) 
   = 2026 \times 100 + 2026 \times 27 
   = 202600 + 54702 
   = 257302.
$$

\noindent 因此,原始总和 \(S = 257302\)。

\noindent 在每次操作中,选 \(x\) 和 \(y\),用 \(x + y - 1\) 取代它们。  
原总和 \(\dots + x + \dots + y + \dots\) 与新总和 \(\dots + (x + y - 1) + \dots\) 的差为
$$
(x + y - 1) - (x + y) = -1.
$$
可见每做一次操作,总和减少 1。

\noindent 共需进行 \(254 - 1 = 253\) 次操作后,总和将减少 \(253\)。  
因此,最终剩下的那个数为:
$$
\text{最终剩下的那个数} 
= 257302 - 253 
= 257049.
$$
   
\newpage

\section{第九题}

\noindent 求解以下方程组在整数解中的解:
$$
x\left(y^2-z^2\right)=2025, \quad y\left(z^2-x^2\right)=341, \quad z\left(x^2-y^2\right)=1664.
$$


\subsection{解答}

\noindent 式一
$$
x\,(y^2 - z^2) \;=\; 2025 \;>\; 0.
$$
说明:
\begin{itemize}
    \item 若 \(x > 0\),则 \(y^2 - z^2 > 0\),即 \(\lvert y\rvert > \lvert z\rvert\);
    \item 若 \(x < 0\),则 \(y^2 - z^2 < 0\),即 \(\lvert y\rvert < \lvert z\rvert\).
\end{itemize}

\noindent 同理,式二
$$
y\,(z^2 - x^2) \;=\; 341 \;>\; 0
$$
说明:
\begin{itemize}
    \item 若 \(y > 0\),则 \(z^2 - x^2 > 0\),即 \(\lvert z\rvert > \lvert x\rvert\);
    \item 若 \(y < 0\),则 \(z^2 - x^2 < 0\),即 \(\lvert z\rvert < \lvert x\rvert\).
\end{itemize}

\noindent 式三
$$
z\,(x^2 - y^2) \;=\; 1664 \;>\; 0
$$
说明:
\begin{itemize}
    \item 若 \(z > 0\),则 \(x^2 - y^2 > 0\),即 \(\lvert x\rvert > \lvert y\rvert\);
    \item 若 \(z < 0\),则 \(x^2 - y^2 < 0\),即 \(\lvert x\rvert < \lvert y\rvert\).
\end{itemize}

\noindent 将三式结合,发现 \(x, y, z\) 要有正有负。

\noindent 先尝试
$$
x < 0, \quad y > 0, \quad z > 0.
$$
那么:
\begin{itemize}
    \item 由式一 \(x(y^2 - z^2) > 0\) 且 \(x < 0\),可知 \(y^2 - z^2 < 0\),即 \(\lvert y\rvert < \lvert z\rvert\)。因 \(y > 0, z > 0\),故 \(y < z\)。
    \item 由式二 \(y(z^2 - x^2) > 0\) 且 \(y > 0\),可知 \(z^2 - x^2 > 0\),即 \(\lvert z\rvert > \lvert x\rvert\)。但 \(x < 0\) 不影响绝对值比较,所以 \(z > \lvert x\rvert\)。
    \item 由式三 \(z(x^2 - y^2) > 0\) 且 \(z > 0\),可知 \(x^2 - y^2 > 0\),即 \(\lvert x\rvert > \lvert y\rvert\)。  
\end{itemize}

\noindent 综合上述分析:
$$
y < |x| < z.
$$

\noindent 因为 \(x < 0\),式一
$$
x\,(y^2 - z^2) \;=\; 2025
$$
等价于
$$
-(|x|)\,(y^2 - z^2) \;=\; 2025,
$$
即
$$
z^2 - y^2 \;=\;\frac{2025}{|x|}.
$$
这意味着 \((z^2 - y^2)\) 是 \(2025\) 的某个除数。同时,\(|x|\) 也必然是 \(2025\) 的正因子。

\noindent \(2025 = 3^4 \times 5^2\),其正因子有:
$$
1, 3, 5, 9, 15, 25, 27, 45, 75, 81, \dots
$$

\noindent 尝试取 \(|x| = 15\),则
$$
z^2 - y^2 = \frac{2025}{15} = 135.
$$
令
$$
x = -15,
$$
式一自动满足。只需解
$$
z^2 - y^2 = 135.
$$

\noindent 将式二化为:
$$
y\,(z^2 - x^2) \;=\; 341 
\quad\Longrightarrow\quad
y\,(z^2 - 225) \;=\; 341.
$$

\noindent 将式三化为:
$$
z\,(x^2 - y^2) \;=\; 1664
\quad\Longrightarrow\quad
z\,(225 - y^2) \;=\; 1664.
$$

\noindent 此时我们有约束:
$$
\begin{cases}
z^2 - y^2 = 135,\\[4pt]
z^2 > y^2\quad (\text{所以 } z > y > 0),\\[3pt]
z > |x| = 15,\quad y < 15.
\end{cases}
$$
将 \(z^2 - y^2 = 135\) 写成因式分解:
$$
(z-y)\,(z+y) = 135.
$$
因 \(z > y > 0\),故 \((z-y)\) 和 \((z+y)\) 是 \(135\) 的正因子对,且 \((z+y) > (z-y)\)。

\noindent \(135\) 的正因子配对为:
$$
135 = 
1 \times 135, \quad 
3 \times 45, \quad
5 \times 27, \quad
9 \times 15.
$$
试着列举:
\begin{enumerate}
    \item \((z-y)=1,\,(z+y)=135\) \(\implies z=68,y=67\);但此时 \(y=67\) 不满足 \(y < 15\),失败。
    \item \((z-y)=3,\,(z+y)=45\) \(\implies z=24,y=21\);此时 \(y=21\) 仍不满足 \(y < 15\),失败。
    \item \((z-y)=5,\,(z+y)=27\) \(\implies z=16,y=11\)。此时 \(z=16 > 15\),\(y=11 < 15\),且 \(z > y > 0\) 成立。
    \item \((z-y)=9,\,(z+y)=15\) \(\implies z=12,y=3\);但此时 \(z=12\) 不满足 \(z > 15\),失败。
\end{enumerate}

\noindent 唯一可行的解为:
$$
z = 16, \quad y = 11.
$$

\noindent 验证式二:
$$
y\,(z^2 - 225) \;=\; 11\,\bigl(16^2 - 225\bigr)
\;=\; 11\,(256 - 225)
\;=\; 11 \times 31
\;=\; 341.
$$
完全满足。

\noindent 验证式三:
$$
z\,(225 - y^2) \;=\; 16\,\bigl(225 - 121\bigr)
\;=\; 16 \times 104
\;=\; 1664.
$$
也完全满足。

\noindent 因此,解为:
$$
x = -15, \; y = 11, \; z = 16.
$$

\newpage

\section{第十题}

\noindent 设 $ABCD$ 是一个凸四边形,面积为 2025,$E$ 是 $AD$ 和 $BC$ 的交点。已知 $\angle AEB = 60^\circ$,且 $AC, BD, AE, BE$ 构成一个递增的等差数列,其公差为 $CE - DE > 0$。求四边形 $ABCD$ 的周长。



\subsection{解答}


\newpage

\end{document}


